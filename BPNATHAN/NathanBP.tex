\documentclass[pdftex,a4paper,12pt]{report}

\usepackage[utf8]{inputenc}  % Accenten gebruiken in tekst (vb. é ipv \'e)
\usepackage{amsfonts}        % AMS math packages: extra wiskundige
\usepackage{amsmath}         %   symbolen (o.a. getallen-
\usepackage{amssymb}         %   verzamelingen N, R, Z, Q, etc.)
\usepackage[dutch]{babel}    % Taalinstellingen: woordsplitsingen,
                             %  commando's voor speciale karakters
                             %  ("dutch" voor NL)
\usepackage{eurosym}         % Euro-symbool €
\usepackage{geometry}
\usepackage{graphicx}        % Invoegen van tekeningen
\usepackage[pdftex,bookmarks=true]{hyperref}
                             % PDF krijgt klikbare links & verwijzingen,
                             %  inhoudstafel
\usepackage{listings}        % Broncode mooi opmaken
\usepackage{multirow}        % Tekst over verschillende cellen in tabellen
\usepackage{rotating}        % Tabellen en figuren roteren
\usepackage{natbib}          % Betere bibliografiestijlen
\usepackage{fancyhdr}        % Pagina-opmaak met hoofd- en voettekst

\usepackage[T1]{fontenc}     % Ivm lettertypes
\usepackage{lmodern}
\usepackage{textcomp}

\usepackage{lipsum}          % Voor vultekst (lorem ipsum)

%%---------- Layout ------------------------------------------------------

% hoofdingen, enz.
\pagestyle{fancy}
% enkel hoofdstuktitel in hoofding, geen sectietitel (vermijd overlap)
\renewcommand{\sectionmark}[1]{}

% lijn, wordt gebruikt in titelpagina
\newcommand{\HRule}{\rule{\linewidth}{0.5mm}}

% Leeg blad
\newcommand{\emptypage}{
\newpage
\thispagestyle{empty}
\mbox{}
\newpage
}

% Gebruik een schreefloos lettertype ipv het "oubollig" uitziende
% Computer Modern
\renewcommand{\familydefault}{\sfdefault}

% Commando voor invoegen Java-broncodebestanden (dank aan Niels Corneille)
% Gebruik: \codefragment{source/MijnKlasse.java}{Uitleg bij de code}
\newcommand{\codefragment}[2]{ \lstset{%
  language=java,
  breaklines=true,
  float=th,
  caption={#2},
  basicstyle=\scriptsize,
  frame=single,
  extendedchars=\true
}
\lstinputlisting{#1}}

%%---------- Documenteigenschappen ---------------------------------------
%% Vul dit aan met je eigen info:

% Je eigen naam
\newcommand{\student}{Nathan Baele}

% De naam van je lector, begeleider, promotor
\newcommand{\promotor}{Bert Van Vreckem}

% De naam van je co-promotor
\newcommand{\copromotor}{Selami Top}

% Indien je bachelorproef in opdracht van een bedrijf of organisatie
% geschreven is, geef je hier de naam.
\newcommand{\instelling}{---}

% De titel van het rapport/bachelorproef
\newcommand{\titel}{Beveiliging van een Windows Server 2012 R2 webserver met ASP.NET applicatie...............}

% Datum van indienen
\newcommand{\datum}{29 mei 2015}

% Faculteit
\newcommand{\faculteit}{Faculteit Bedrijf en Organisatie}

% Soort rapport
\newcommand{\rapporttype}{Scriptie voorgedragen tot het bekomen van de graad van\\Bachelor in de toegepaste informatica}

% Academiejaar
\newcommand{\academiejaar}{2014-2015}

% Examenperiode
%  - 1e semester = 1e examenperiode
%  - 2e semester = 2e examenperiode
%  - tweede zit = 3e examenperiode
\newcommand{\examenperiode}{Tweede examenperiode}

%%========================================================================
%% Inhoud document
%%========================================================================

\begin{document}

%%---------- Front matter ------------------------------------------------
%% Het voorblad - Hier moet je in principe niets wijzigen.

\begin{titlepage}
  \newgeometry{top=2cm,bottom=1.5cm,left=1.5cm,right=1.5cm}
  \begin{center}

    \begingroup
    \rmfamily
    \includegraphics[width=2.5cm]{img/HG-beeldmerk-woordmerk}\\[.5cm]
    \faculteit\\[3cm]
    \titel
    \vfill
    \student\\[3.5cm]
    \rapporttype\\[2cm]
    Promotor:\\
    \promotor\\
    Co-promotor:\\
    \copromotor\\[2.5cm]
    Instelling: \instelling\\[.5cm]
    Academiejaar: \academiejaar\\[.5cm]
    \examenperiode
    \endgroup

  \end{center}
  \restoregeometry
\end{titlepage}

% Schutblad

\emptypage


\begin{titlepage}
  \newgeometry{top=5.35cm,bottom=1.5cm,left=1.5cm,right=1.5cm}
  \begin{center}

    \begingroup
    \rmfamily
    \faculteit\\[3cm]
    \titel
    \vfill
    \student\\[3.5cm]
    \rapporttype\\[2cm]
    Promotor:\\
    \promotor\\
    Co-promotor:\\
    \copromotor\\[2.5cm]
    Instelling: \instelling\\[.5cm]
    Academiejaar: \academiejaar\\[.5cm]
    \examenperiode
    \endgroup

  \end{center}
  \restoregeometry
\end{titlepage}


\begin{abstract}
% TODO: De "abstract" of samenvatting is een kernachtige (max 1 blz. voor een
% thesis) synthese van het document. In ons geval beschrijf je kort de
% probleemstelling en de context, de onderzoeksvragen, de aanpak en de
% resultaten.
Vandaag de dag hoor je regelmatig eens in het nieuws dat er een bedrijf is opgelicht door professionele hackers, oplichters die zijn binnen gedrongen in hun netwerk en gevoelige informatie hebben gebruikt om zaken te verkrijgen. Dit probleem groeit even snel als de groei van netwerken in het bedrijfsleven. Daarom is het belangrijk om een zeer goed beveiligd netwerk te hebben tegen bedreigingen van zowel binnen als buiten het bedrijf. \newline \newline

Mijn grootste doelstelling is om een overzicht te voorzien van welke soorten maatregelen er zeker moeten getroffen worden om een netwerk optimaal te beveiligen. Dit gaande van de router tot de switch tot de server. Ik wil zelf ook zo een beveiligd netwerk/server kunnen opzetten en zelf kunnen testen dat er geen enkele vorm van bekende bedreigingen binnen kan. Tot slot wil ik te weten komen of er in het bedrijfsleven wel nood en budget is voor zulke hevige beveiligingen. \newline \newline

Om dit probleem te onderzoeken heb ik op voorhand enkele onderzoeksvragen vastgesteld. Wat zijn de bekendste soorten van externe en interne bedreigingen en hoe worden deze het efficiëntst opgelost? Hoe word je router en switch zo optimaal mogelijk beveiligd? Hoe wordt de server zo goed mogelijk beveiligd? Wat zijn de voor -en nadelen van bepaalde beveiligingstechnieken? \newline 


\textbf{Zijn de 'best practices' voldoende als beveiliging tegen een externe of interne aanval? Wat is de beste manier om als administrator sporen terug te vinden van een aanval?
} (KAN NOG VERANDEREN)
\end{abstract}

\chapter*{Voorwoord}
\label{ch:voorwoord}
Deze scriptie zou niet to stand gekomen zijn zonder de hulp van mijn stagementor en co-promotor Selami Top. Ik mocht gebruik maken van zijn huidig netwerk en ik mocht enkele zaken uitproberen op zijn nieuw netwerk. Hierdoor kon ik de zaken die ik onderzocht en opgezocht had uit proberen in een echte omgeving en kreeg ik een betere kijk op een realistische beveiliging. Verder wil ik ook mijn promotor Bert Van Vreckem bedanken die mij heeft geholpen om deze bachelorproef tot stand te brengen. Zijn tips en technische kennis waren een enorme hulp om dit resultaat te bekomen. (NOG WAT TOEVOEGEN). Tot slot wil ik alle auteurs bedanken van de lectuur die ik heb gebruikt om deze scriptie te maken (LIJST VAN ALLE AUTEURS?)
% TODO: Vergeet ook niet te bedankten wie je geholpen/gesteund/... heeft

\tableofcontents

% Als je een lijst van afkortingen of termen wil toevoegen, dan hoort die
% hier thuis. Gebruik bijvoorbeeld de ``glossaries'' package.

%%---------- Kern --------------------------------------------------------

\chapter{Inleiding}
\label{ch:inleiding}

De inleiding moet de lezer alle nodige informatie verschaffen om het onderwerp te begrijpen zonder nog externe werken te moeten raadplegen \citep{Pollefliet2011}. Dit is een doorlopende tekst die gebaseerd is op al wat je over het onderwerp gelezen hebt (literatuuronderzoek). \newline \newline

Je verwijst bij elke bewering die je doet, vakterm die je introduceert, enz. naar je bronnen. In \LaTeX{} kan dat met het commando \texttt{$\backslash${cite\{\}}} of \texttt{$\backslash${citep\{\}}}. Als argument van het commando geef je de ``sleutel'' van een ``record'' in een bibliografische databank in het Bib\TeX{}-formaat (een tekstbestand). Als je expliciet naar de auteur verwijst in de zin, gebruik je \texttt{$\backslash${}cite\{\}}.
Soms wil je de auteur niet expliciet vernoemen, dan gebruik je \texttt{$\backslash${}citep\{\}}. Hieronder een voorbeeld van elk.

\cite{Knuth1998} schreef een van de standaardwerken over sorteer- en zoekalgoritmen. Experten zijn het erover eens dat cloud computing een interessante opportuniteit vormen, zowel voor gebruikers als voor dienstverleners op vlak van informatietechnologie~\citep{Creeger2009}.

\section{Probleemstelling en Onderzoeksvragen}
\label{sec:onderzoeksvragen}

% TODO: Wees zo concreet mogelijk bij het formuleren van je
% onderzoeksvra(a)g(en). Een onderzoeksvraag is trouwens iets waar nog
% niemand op dit moment een antwoord heeft (voor zover je kan nagaan).
\subsection{Zijn de 'best practices' voldoende als beveiliging tegen een externe of interne aanval?}

Port scanning is een bekende manier om iemand zijn netwerk in kaart te brengen en te kijken naar een manier hoe je er binnen kan geraken. Port scanning is kan jouw netwerk in gevaar brengen, maar kan er ook voor zorgen dat jouw netwerk beter beveiligd is als je weet hoe je het moet gebruiken en hoe je jezelf ertegen moet beschermen. De vraag die hier wordt gesteld is hoe dat je port scanning als ethische hacker kan gebruiken om jouw netwerk beter te beveiligen tegen mensen die port scannen gebruiken voor niet zo ethische doelstellingen.

\subsection{Wat is de beste manier om als administrator sporen terug te vinden van een aanval?}

Het spreekt voor zich dat, wanneer er zich een aanval voordoet of heeft voorgedaan, dat je als netwerkbeheerder dit direct of toch zo snel mogelijk wilt te weten komen. Als er een aanval gaande is dan is het belangrijk dat je dit snel weet en dat je snel de oorzaak vindt en weet wat er precies aan het gebeuren is. Hetzelfde geldt voor wanener er een aanval heeft plaatsgevonden in de geschiedenis. Als administrator is het jouw taak om dit makkelijk op te sporen en ervoor te zorgen dat dit in de toekomst niet meer gebeurd.

\chapter{Methodologie}
\label{ch:methodologie}

% TODO: Hoe ben je te werk gegaan? Verdeel je onderzoek in grote fasen, en
% licht in elke fase toe welke stappen je gevolgd hebt. Verantwoord waarom je
% op deze manier te werk gegaan bent. Je moet kunnen aantonen dat je de best
% mogelijke manier toegepast hebt om een antwoord te vinden op de
% onderzoeksvraag.
\section{Voorbereiding}
\subsection{Onderzoek}
Om een goed antwoord te kunnen geven op deze onderzoeksvraag moet ik eerst een goede kennis hebben over het onderwerp. Dit heb ik gedaan door het verzamelen van lectuur over de volgende zaken:
\begin{itemize}
	\item Ethisch hacken
	\item Beveiligen van Windows Server 2012 R2-netwerk best practices
\end{itemize}
Deze lectuur gaf mij een degelijke basis om mijn onderzoek te starten. In de bibliografie is te zien welke lectuur ik hier specifiek voor heb gebruikt. Natuurlijk kan je met te lezen niet alles leren dus heb ik ook enkele zaken moeten oefenen. Ik ben begonnen bij het begin en heb eerst een oude Windows XP-machine zonder enkele patch en firewall proberen te hacken. In theorie is dit niet moeilijk, maar je moet ergens beginnen. Ik heb mijn Kali-machine opgestart en heb eerst een port scan uitgevoerd om te kijken welke poorten er open waren. Nadat ik dit wist heb ik gezocht naar een exploit, heb ik deze gevonden en ben ik via deze exploit binnen geraakt in de opdrachtprompt van de Windows XP-machine met administrator rechten. Door deze kleine oefening leerde ik de methodiek die wordt gebruikt om in werkstations en servers binnen te dringen. \newline \newline

Natuurlijk was voorgaande oefening enorm simpel, maar deze was dan ook voor de techniek onder de knie te krijgen. Nu is het tijd voor een iets moeilijkere oefening.

\subsection{Opzetten testomgeving}
Om zelf wat ervaring op te doen met port scanning ben ik gestart met een virtuele machine te maken waarop Windows Server 2012 R2 op geïnstalleerd staat en deze heb ik de naam \textbf{ADServer} genoemd. Als eerste heb ik deze domeincontroller gemaakt in het fictieve domein \textit{Baele.be}. Verder heb ik ook de rollen DNS, DHCP en Externe toegang geïnstalleerd en heb ik als subnet gekozen voor 192.168.1.0/24 waar ik de range 192.168.1.1 tot 192.168.1.30 heb ik uitgesloten voor distributie. Dan ben ik naar de website gegaan van Microsoft om SQL server 2012 te downloaden en te installeren. Ook heb ik de rol IIS 8 geïnstalleerd en heb ik een kleine lokale website gemaakt met een ASP.NET-applicatie op, deze is te bereiken via "\textit{baele.be}". Deze applicatie heb ik gemaakt via Microsoft Visual studio en heeft ook een SQL-databank in de backend. Op de startpagina worden namen weergegeven die in een tabel in de databank zitten. Je kan je ook inloggen via een naam en wahtwoord die zich in de databank bevinden. Als je bent ingelogged dan kan je een speciale pagina "member" bekijken en als je niet ingelogged bent dan lukt dit niet. Tot slot heb ik de server het IP-adres 192.168.1.2 gegeven. Deze zaken moeten zich allemaal op de server bevinden zodat ik mijn onderzoeksvraag goed kan beantwoorden. \newline

Nu dat er 1 server opstaat is het tijd om enkele hosts op te zetten en deze toe te voegen aan het domein. Ik heb 3 Windows hosts opgezet met de IP-adressen 192.168.1.31 - 192.168.1.32 - 192.168.1.33 en naam WS1 - WS2 - WS3. WS1 is een Windows 8.1-machine, WS2 is een Windows 7-machine en WS1 is een Windows Vista-machine. Tot slot heb ik een kali virtuele machine gemaakt om in het netwerk binnen te dringen. Met deze 5 virtuele machines kan ik verschillende soorten software en technologieën testen die ervoor zorgen dat ik mijn onderzoeksvragen zo nauwkeurig en correct mogelijk kan beantwoorden. 

\subsection{}
%% TODO: de structuur en titel van deze hoofdstukken hangen af van je
% eigen onderzoek. Elke fase in je onderzoek kan een eigen hoofdstuk krijgen. Kies telkens een gepaste titel. ``Corpus'' is *GEEN* gepaste titel

\chapter{Vastlopen van de server}
Het vastlopen van een server kan op verschillende manieren. De eerste manier die ik heb geprobeerd om de webserver te laten vastlopen is \textit{sockstress}. Dit is een methode die wordt gebruikt om servers over het internet aan te vallen gebruik makende van TCP. Deze methode zorgt ervoor dat het lokale geheugen zoveel aanvragen moet behandelen dat deze langzaam maar zeker helemaal volloopt totdat de server is vastgelopen en de server is gecrashed. Je kan dit ook een DOS (Denial of Service)-aanval noemen.


\chapter{Conclusie}
\label{ch:conclusie}

% TODO: Trek een duidelijke conclusie, in de vorm van een antwoord op de
% onderzoeksvra(a)g(en). Reflecteer kritisch over het resultaat. Zijn er
% zaken die nog niet duidelijk zijn? Heeft het ondezoek geleid tot nieuwe
% vragen die uitnodigen tot verder onderzoek?
\lipsum[76-80]


\bibliographystyle{apa}
\bibliography{bibliografie BP}

%%---------- Back matter -------------------------------------------------

\listoffigures
\listoftables

\end{document}
